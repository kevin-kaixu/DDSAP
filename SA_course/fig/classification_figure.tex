\begin{table}[t!]
\small
  \centering
    \begin{tabular*}{0.48\textwidth}{l|c|c|c|c}
    \hline
    Method                & Input Data  & Shapes & Classes & Acc         \\
    \hline
    \hline
    \protect\cite{Huang:2013:FSL} & 3D Warehouse & 1206-5850 & 9-26 & 86 \\
    \protect\cite{Golovinskiy:2009:SBR3D} & LIDAR & 1063  & 16 & 58  \\
    \protect\cite{Shilane:2007:drs} & PSB & 1814 & 90 & 75 \\
    \protect\cite{Funkhouser:2006:pm3d} & PSB & 1814 & 90 & 83 \\
    \protect\cite{Barutcuoglu:2006:hscu} & PSB & 1814 & 90 & 84 \\
    \protect\cite{Bronstein:2011:SGGW} & SG & 715 & 13 & 89 \\
    \protect\cite{Litman:2014:SLBF} & SG & 715 & 13 & 91 \\
    \protect\cite{Li:2012:SHREC} & SHREC12 & 1200 & 60 & 88 \\
     \protect\cite{Li:2014:SHREC} & SHREC14 & 400-$10^4$ & 19-352 & 87\\
%    \hline
%    \multicolumn{5}{c}{Nearest neighbor classifier with different shape descriptors} \\
%    \hline 
%    \cite{Chaudhuri:2013:ACC} & supervised     & pairwise comparison  & meshes (parts) & 42-100 & 7-14 & N/A \\
    \hline
    \end{tabular*}
  \caption{\rev{Performance of several methods for shape classification (the accuracy in the right-most column as measured as fraction of correctly-labeled shapes). Huang et al.~\protect\cite{Huang:2013:FSL}
predict fine-grained tag attributes for big collections of similar shapes. Golovinskiy et al.~\protect\cite{Golovinskiy:2009:SBR3D}
propose a method for classifying point clouds of objects in urban environments. The methods aimed at classifying meshes are evaluated on Princeton Shape Benchmark (PSB)~\protect\cite{Shilane:2007:drs,Funkhouser:2006:pm3d,Barutcuoglu:2006:hscu}. To evaluate performance of the method in the presence of non-rigid deformations ShapeGoogle (SG) dataset is also commonly used~\protect\cite{Bronstein:2011:SGGW,Litman:2014:SLBF}.
In addition several methods can be found in regular shape retrieval challenges. The winner of 2012 challenge~\protect\cite{Li:2012:SHREC} is ~\protect\cite{Ohbuchi:2010:DML} and the winner of 2014 challenge~\protect\cite{Li:2014:SHREC} is DBSVC technique, both methods use bag of features for classification. } }
  \label{tab:classification_performance}
\end{table}




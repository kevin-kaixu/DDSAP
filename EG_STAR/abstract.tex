Data-driven methods serve an increasingly important role in discovering geometric, structural, and semantic relationships between shapes. 
In contrast to traditional approaches that process shapes in isolation of each other, data-driven methods aggregate information from 3D model collections to improve the analysis, modeling and editing of shapes.
\rev{Data-driven methods are also able to learn computational models that reason about properties and relationships of shapes without relying on hard-coded rules or explicitly programmed instructions.} 
Through reviewing the literature, we provide an overview of the main concepts and components of these methods, as well as discuss their application to classification, segmentation, matching, reconstruction, modeling and exploration, as well as scene analysis and synthesis.  
We conclude our report with ideas that can inspire future research in data-driven shape analysis and processing. 